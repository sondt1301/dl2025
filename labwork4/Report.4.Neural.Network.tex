\documentclass{article}
\usepackage{multirow}
\usepackage{graphicx}

\title{Labwork 4: Neural Network}
\author{Dang Thai Son}
\date{May 2025}

\begin{document}

\maketitle

\section{Network Architecture}
Using proper object-oriented concepts, the network has three classes representing its three main components: 

\begin{itemize}
    \item Neuron class: represents a node in the network. This class has information about weights and inputs which are used to calculate the linear sum and sigmoid value for the node in the next layer.
    \item Layer class: represents a layer in the network. This class defines the number of neurons in each layer and forms a matrix of nx1 neurons based on weights and inputs.
    \item Network class: represents the whole neural network. This class defines the number of layers in the network and applies the feedforward technique to obtain the output of the network.
\end{itemize}

The feedforward algorithm is implemented in the Network class where it uses neurons in the previous layer to calculate those in the next layer using the activation function:
\[
\hat{y} = \sigma\left(-\left(w_1 x_i^{(1)} + w_2 x_i^{(2)} + w_0\right)\right)
\]

\section{Experiment with the XOR network}

\begin{itemize}
    \item With the following fixed weight:
        \begin{center}
            \begin{tabular}{ |c|c|c|c| } 
                \hline
                Layer & x1 & x2 & bias \\
                \hline
                \multirow{2}{4em}{Layer 1} & -1 & -1 & 1.5 \\ 
                & 1 & 1 & -0.5 \\ 
                \hline
                \hline
                \multirow{1}{4em}{Layer 2} & 1 & 1 & -1.5 \\ 
                \hline
            \end{tabular}
        \end{center}
    the result (which is correct) is:
        \begin{center}
            \begin{tabular}{ |c|c|c| } 
                \hline
                x1 & x2 & y \\
                \hline
                0 & 0 & 0 \\ 
                0 & 1 & 1 \\ 
                1 & 0 & 1 \\ 
                1 & 1 & 0 \\ 
                \hline
            \end{tabular}
        \end{center}
    
    \item With random weight:
        \begin{center}
            \begin{tabular}{ |c|c|c|c| } 
                \hline
                Layer & x1 & x2 & bias \\
                \hline
                \multirow{2}{4em}{Layer 1} & 0.64 & 0.28 & 0.03 \\ 
                & 0.33 & 0.46 & 0.06 \\ 
                \hline
                \hline
                \multirow{1}{4em}{Layer 2} & 0.52 & 0.91 & 0.09 \\ 
                \hline
            \end{tabular}
        \end{center}
    the result (which is not correct and need further update) is:
        \begin{center}
            \begin{tabular}{ |c|c|c| } 
                \hline
                x1 & x2 & y \\
                \hline
                0 & 0 & 1 \\ 
                0 & 1 & 1 \\ 
                1 & 0 & 1 \\ 
                1 & 1 & 1 \\ 
                \hline
            \end{tabular}
        \end{center}
\end{itemize}
\end{document}
