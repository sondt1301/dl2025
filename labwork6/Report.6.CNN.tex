\documentclass{article}
\usepackage{graphicx}
\usepackage{amsmath}
\usepackage{amsfonts}

\title{Labwork 6: CNN}
\author{Dang Thai Son}
\date{May 2025}

\begin{document}

\maketitle

\section{Network Architecture}

The VGG19 model comprises two main sequential modules:

\begin{enumerate}
    \item \textbf{Feature Extractor ($f_{\text{features}}$)}:
    processes an input image to extract hierarchical features:
    \begin{itemize}
        \item Multiple Convolutional layers: Each layer applies $N_k$ filters of size $3 \times 3$:
        \[ Y_{conv} = \text{ReLU}(W * X' + b) \]
        where $X'$ is the input feature map to the layer, $W$ are filter weights, $b$ is the bias.
        \item Max Pooling layers (`M`): A $2 \times 2$ max-pooling operation.
    \end{itemize}

    \item \textbf{Classifier ($f_{\text{classifier}}$)}:
    The output $X_{feat}$ from the feature extractor is first flattened into a vector $X_{flat} = \text{Flatten}(X_{feat})$. This vector is then processed by a sequence of fully connected layers:
    \begin{itemize}
        \item Hidden layers:
        \[ Y_{fc} = \text{ReLU}(W_{fc} X_{flat}' + b_{fc}) \]
        where $X_{flat}'$ is the input vector to the layer.
        \item Output layer: A final fully connected layer with $N_{\text{classes}}$ units producing the raw output scores $Z_{out}$:
        \[ Z_{out} = W_{out} X_{hidden} + b_{out} \]
    \end{itemize}
    The overall classification is $Z_{out} = f_{\text{classifier}}(X_{flat})$.
\end{enumerate}

\end{document}
